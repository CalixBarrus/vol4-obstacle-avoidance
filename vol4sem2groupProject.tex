% This creates a document based on the AMS
% (American Mathematical Society) article style.
\documentclass[oneside,12pt]{amsart}

% This package simply sets the margins to be 1.00 inch.
% If you prefer 1.25 in margins change the value.
% If you look up the documentation for the geometry package
% you can find out how to control the size of all four margins
% independently of each other.
\usepackage[margin=1.00in]{geometry}

% These packages include nice commands from AMS-LaTeX
% AMS-LaTeX has lots of nice equation environment that are better
% than the plan LaTeX options.
\usepackage{amssymb,amsmath,amsthm, enumitem, bm, graphicx, mathrsfs, bbm}

% Make the space between lines slightly more
% generous than normal single spacing, but compensate
% so that the spacing between rows of matrices still
% looks normal.  Note that 1.1=1/.9090909...
\renewcommand{\baselinestretch}{1.1}
\renewcommand{\arraystretch}{.91}
% If you want even more space between lines try
% \renewcommand{\baselinestretch}{1.5}
% \renewcommand{\arraystretch}{0.67}

% Define a new environment for exercises.
\newenvironment{exercise}[1]{\vspace{.1in}\noindent\textbf{Exercise #1 }}{}

% Define shortcut commands for commonly used symbols.
\newcommand{\R}{\mathbb{R}} % Real numbers
\newcommand{\C}{\mathbb{C}} % Complex numbers
\newcommand{\Z}{\mathbb{Z}} % Integers
\newcommand{\Q}{\mathbb{Q}} % Rational numbers
\newcommand{\N}{\mathbb{N}} % Natural numbers
\newcommand{\calP}{\mathcal{P}} % Calligraphic P for power set
\newcommand{\calN}{\mathcal{N}}
\newcommand{\calR}{\mathcal{R}}
\newcommand{\F}{\mathbb{F}} % Field
\newcommand{\E}{\mathbb{E}} % Expected Value

\newcommand{\unif}{\text{Uniform}} 
\newcommand{\normaldist}{\mathcal{N}} 
\newcommand{\gammadist}{\text{Gamma}} 
\newcommand{\empirical}{\text{Emp}} 
\newcommand{\argmax}{\text{ArgMax}} 
\newcommand{\argmin}{\text{ArgMin}} 
\newcommand{\pdf}{f} 
\newcommand{\given}{|} 
\newcommand{\KL}{\mathbb{KL}} 

% vectors and subspaces
\renewcommand{\vec}[1]{{\ensuremath{\bm{#1}}}}
\newcommand{\0}{{\vec  0 }}
\newcommand{\1}{{\mathbbm{  1} }}
\newcommand{\J}{{\vec  J }}
\renewcommand{\a}{{\vec  a }}
\renewcommand{\b}{{\vec  b }}
\renewcommand{\c}{{\vec  c }}
\renewcommand{\d}{{\vec  d }}
\newcommand{\e}{{\vec  e }}
\newcommand{\f}{{\vec  f }}
\newcommand{\g}{{\vec  g }}
\newcommand{\h}{{\vec  h }}
\renewcommand{\k}{{\vec  k }}
\renewcommand{\l}{{\boldsymbol{\ell} }}  % lowercase l looks too much like 1.  
\newcommand{\m}{{\vec  m }}
\newcommand{\p}{{\vec  p }}
\newcommand{\q}{{\vec  q }}
\renewcommand{\r}{{\vec  r }}
\newcommand{\s}{{\vec  s }}
\renewcommand{\t}{{\vec  t }}
\let\uaccent\u
\renewcommand{\u}{{\vec  u }}
\let\vaccent\v
\renewcommand{\v}{{\vec  v }}
\newcommand{\w}{{\vec  w }}
\newcommand{\x}{{\vec  x }}
\newcommand{\y}{{\vec  y }}
\newcommand{\z}{{\vec  z }}
\newcommand{\bZ}{{\vec{Z}}}
\newcommand{\X}{{\mathbb{X}}}
\renewcommand{\L}{{\mathscr{L}}}

% Define a function called "span" that would be useful in linear algebra.
\DeclareMathOperator{\vsspan}{span}

% Define functions for the real and imaginary parts of a complex number.
% These could be used in place of the built in commands \Re and \Im,
% which print Re and Im in a font that some people don't like.
\DeclareMathOperator{\re}{Re}
\DeclareMathOperator{\im}{Im}



%%%%%%%%%%%%%%%%%%%%%%%%%%%%%%%%%%%%%%%%%%

\begin{document}
    
    % If you use Overleaf, the name of the project will be determined by
    % what you enter as the document title.
    \title{Vol 4 Project }
    
    \begin{center}
        \textsf{Problem Statement and Solution Ideas}
    \end{center}
    
    Group Members:
    Dallan Olsen, Laren Edwards, Riley Gabrielson, Calix Barrus
    
    \section{Premise}
    Our foundation is a robot with 4 independent wheels, and our goal is to solve optimal avoidance paths using the 4 wheels as our control. We feel that this extension to the standard object avoidance problem provides more direct application to real world robotics problems.
    
    \section{Cost Function}
    We have for our cost functional the equation 
    J[u] =0tfc1 ds+ 0tfc2(x''2 + y''2) + c3 C(x, y)dt +c4tf 
    Where C(x,y) is given as in the Obstacle avoidance lab. This cost functional allows us to penalize time, path length, acceleration, as well as the ability to impose a stiff penalty for colliding with (or getting too close to) the obstacle. 
    
    \section{State Equations}
    The evolution of our state is given by
    
    \begin{align*}
        \vec{x}’ = H \begin{bmatrix} x \\ y \\ \theta \\ x' \\ y' \\ \theta' \end{bmatrix}   + F \begin{bmatrix} 0 \\ 0 \\ 0 \\  \vec{\phi}(\vec{u})  \end{bmatrix} \\
    \vec{x}(t_f) = [x_f, y_f, \ldots]^T
    \end{align*}
    
    Where 
    \begin{align*}
        H = 
        \begin{bmatrix} 
            0 & 0 & 0 & 1 & 0 & 0\\  
            0 & 0 & 0 & 0 & 1 & 0\\  
            0 & 0 & 0 & 0 & 0 & 1\\  
            0 & 0 & 0 & 0 & 0 & 0\\  
            0 & 0 & 0 & 0 & 0 & 0\\  
            0 & 0 & 0 & 0 & 0 & 0\\  
        \end{bmatrix},
    F =
    \begin{bmatrix} 
        0 & 0 & 0 & 0 & 0 & 0 \\  
        0 & 0 & 0 & 0 & 0 & 0 \\  
        0 & 0 & 0 & 0 & 0 & 0 \\  
        0 & 0 & 0 & 1 & 0 & 0 \\  
        0 & 0 & 0 & 0 & 1 & 0 \\  
        0 & 0 & 0 & 0 & 0 & 1 \\  
    \end{bmatrix} 
    \end{align*}
    
    This basically states that first derivatives are related to themselves and the second derivatives are controlled by the control variable. 
    
    The phi function is a result of solving the equation of motion, and describes how the 4 controls (motors 1,2,3,4) influence x’, y’, and theta’. In order to solve this problem with LQR, we may need to linearize phi.
    
    \section{Plan to Solve}
    \begin{enumerate}[label=\roman*.]
        \item Derive and solve equations of motion to get $\phi$.
        \item Solve for optimal path using LQR or a numerical scheme. This will involve making coming up with the LQR equations and solving for the optimal path.
        \item Come up with jupyter code and plots of states and controls.
        \item Plug situation into physics engine.
        \begin{enumerate}[label=\alph*.]
            \item Use PID to stick to optimal path in engine.
        \end{enumerate}
        \item Attempt recreate simulation with physical robots.
        \begin{enumerate}[label=\alph*.]
            \item Work out how the sensors work and getting them to feed info to our code. 
        \end{enumerate}
    \end{enumerate}
    
    
    
    %---------------------------------
    % Don't change anything below here
    %---------------------------------
    
    
\end{document}
